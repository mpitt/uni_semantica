%        File: appunti.tex
%    	Created: Mon Feb 24 02:00 PM 2014 C
% 	Last Change: Sat 08 Mar 18:00 PM 2014 C by Leo
%
\documentclass[a4paper]{report}
\usepackage[utf8]{inputenc}
\usepackage[italian]{babel}
\usepackage{amsmath}
\usepackage{amssymb}
\usepackage{amsthm}
\usepackage{relsize}
\usepackage[retainorgcmds]{IEEEtrantools}

\newcommand{\bnfn}{\ \mathrm{b.f.}\ }
\newcommand{\sse}{se e solo se}
\newcommand{\tc}{\ \mathrm{t.c.}\ }
\newcommand{\prop}{\mathcal{P}}
\newcommand{\base}[1]{\mathtt{base}_{#1}}
\newcommand{\bop}{\ \mathrm{bop}\ }
\newcommand{\uop}{\mathrm{uop}\ }
\newcommand{\FI}{\mathrm{FI}}
\newcommand{\BI}{\mathrm{BI}}
\newcommand{\IDE}{\mathrm{IDE}}
\newcommand{\perhpind}{\ \stackrel{\text{HP IND}}{\Longrightarrow}\ }
\newcommand{\nil}{\mathrm{nil}}
\newcommand{\const}{\mathrm{const}\ }
\newcommand{\var}{\mathrm{var}\ }
\newcommand{\Dval}{\mathrm{DVal}}
\newcommand{\Dtyp}{\mathrm{DTyp}}
\newcommand{\Loc}{\mathrm{Loc}}
\newcommand{\loc}{\ \mathrm{loc}}

\theoremstyle{definition} \newtheorem*{defi}{Def}
\theoremstyle{plain} \newtheorem{lemma}{Lemma}
\theoremstyle{plain} \newtheorem{teo}{Teorema}
\theoremstyle{remark} \newtheorem*{es}{Esempio}

\DeclareRobustCommand{\frac}[3][0pt]{%
  {\begingroup\hspace{#1}#2\hspace{#1}\endgroup\over\hspace{#1}#3\hspace{#1}}}

\begin{document}
\chapter{Induzione strutturale}
\section*{Introduzione}
Sia $A$ un insieme, sia $\lhd$ una relazione binaria definita su $A$
($\lhd : A \times A$).
\begin{defi}
  $\lhd$ è $\bnfn$ (ben fondata) se $\nexists \ldots \lhd a_i \lhd \ldots \lhd a_1 \lhd a_0$
  \\
  ovvero non esistono catene infinite discendenti.
  \\
  $(A, \lhd)$ è un insieme ben fondato se $\lhd$ è $\bnfn$
\end{defi}
Sia $\unlhd$ la chiusura riflessiva e transitiva di $\lhd \ \bnfn$ su $A$.
\begin{lemma}
  $\unlhd$ non è mai $\bnfn$
\end{lemma}
\begin{proof}
  Sia $a_i \in A$. $\exists \ldots a_i \unlhd a_i \unlhd \ldots$ \qedhere
\end{proof}
\begin{defi}
  Sia $a \in A$. $a$ è \emph{minimale} in $A$ rispetto a $\lhd$ se
  $\forall b \lhd a, b \not\in A$.
\end{defi}
\begin{lemma}
  \label{lemma:2}
  $\lhd$ è ben fondato su $A$ \sse{} ogni $B \subseteq A$ ha un elemento
  minimale rispetto a $\lhd$.
\end{lemma}
\begin{proof}
  \begin{itemize}
    \item{$\Rightarrow$)} Da $ B \subseteq A$ e $(A, \lhd) \bnfn$ si ha che
      non esiste $\ldots \lhd b_i \lhd \ldots \lhd b_0$, quindi
      $\exists b_n \tc b_n \lhd \ldots \lhd b_i \lhd \ldots \lhd b_0$
      ovvero $b_n$ è minimale in $B$ rispetto a $\lhd$.
    \item{$\Leftarrow$)} Per assurdo. Supponiamo che esista
      $\ldots \lhd a_i \lhd \ldots \lhd a_0$, ovvero $\neg (\lhd \bnfn)$.
      Allora l'insieme $B=\{a_i|i\in\mathbb{N}\}$ (insieme degli elementi
      della sequenza) non ha un minimale, cosa che contraddice l'ipotesi.
      Quindi una tale sequenza non esiste, ovvero $\lhd \bnfn$
  \end{itemize}
\end{proof}
\newpage

\section{Principio di induzione noetheriana}
\begin{teo}[Principio di induzione noetheriana (prima forma)]
  Sia $\prop$ una proprietà su $(A, \lhd) \bnfn$.\\
  \[
    \forall a \in A.\prop(a) \Leftrightarrow
    \forall a \in A.([\forall b \lhd a . \prop(b)]
    \Rightarrow \prop(a))
  \]
\end{teo}
\begin{proof}
  \begin{itemize}
    \item{$\Rightarrow$)} Ovvia.
    \item{$\Leftarrow$)} Per assurdo.\\
      Supponiamo
      \begin{equation}
        \forall a \in A.([\forall b \lhd a.\prop(b)]
        \Leftarrow \prop(a))
      \end{equation}
      e
      \begin{equation}
        \exists c \in A .\neg\prop(c)
      \end{equation}
      Sia $C=\{c \in A | \neg \prop(c)\} \subseteq A$.\\
      Per il lemma \ref{lemma:2}, $\exists \hat{c}$ minimale di $C$ rispetto a
      $\lhd$, allora $\neg \prop(\hat{c})$\\
      Per $\hat{c}$ minimale, $\forall b \lhd \hat{c}.b \not\in C$, allora
      $\prop(b)$, ovvero per (1.1) $\prop(\hat{c})$
  \end{itemize}
\end{proof}

\begin{defi}
  $\base{A}=\{a \in A | a$ è minimale in $A$ rispetto a $\lhd \}$.\\
  Osserviamo che $\forall a \in \base{A}, \forall b \in A . b \ntriangleleft a$.
\end{defi}

\begin{teo}[Principio di induzione noetheriana - seconda forma]
  Sia $\prop$ una proprietà su $(A, \lhd) \bnfn$.\\
  \[
    \forall a \in A.\prop(a) \Leftrightarrow
    \begin{pmatrix}
      \forall a \in \base{A}.\prop(a)\\
      \wedge\\
      \forall a \in (A \setminus \base{A})
      .([\forall b \lhd a . \prop(b)] \Rightarrow \prop(a))
    \end{pmatrix}
  \]
\end{teo}

\begin{teo}[Induzione matematica]
  Sia $A=\mathbb{N}$.\\
  Sia $n \lhd m \Leftrightarrow m=n+1 \quad n,m \in \mathbb{N}$.\\
  Osserviamo che $\lhd$ è ben fondata: $0\lhd1\lhd2\lhd\dots$.\\
  Osserviamo $\base{\mathbb{N}}=\{0\}$.
  \[
    \forall m \in \mathbb{N}.\prop(m) \Leftrightarrow
    \begin{pmatrix}
      \prop(0)\\
      \wedge\\
      \forall m > 0
      .(\prop(m-1) \Rightarrow \prop(m))
    \end{pmatrix}
  \]
\end{teo}

\begin{defi}
  $a^\lhd=\{b \in A|b \lhd a\}$ con $(A,\lhd)\bnfn$.
\end{defi}
\begin{defi}
  Sia $f: A \rightarrow B$, sia $A' \subseteq A$.
  $f_{|A'}=\{(a, f(a))|a \in A'\}$.
\end{defi}

\begin{teo}[di ricorsione / delle definizione noetheriane]
  Sia $(A,\lhd)\bnfn$
  \[
    \forall a \in A, \forall h:a^\lhd \rightarrow B.
    F(a,h) \in B
  \]
  ($F$ è detta \emph{operatore di composizione}). Allora
  \[
    \exists! f: A \rightarrow B \tc
    \forall a \in A.f(a)=F(a,f_{|a^\lhd})
  \]
\end{teo}
\begin{es}
  Sia $A=\mathbb{N}$, sia $n \lhd m \Leftrightarrow m=n+1$.\\
  \[
    \mathtt{Fact}(n)=
    \begin{cases}
      1 & \text{se } n=0\\
      n \cdot \mathtt{Fact}(n-1) & \text{se } n>0
    \end{cases}
  \]
  In questo caso, $f = \mathtt{Fact}$ e $F$ è la moltiplicazione.\\
  $f(n)=F(n,f(n-1))=n \cdot f(n-1)$
\end{es}
\newpage{}
\begin{teo}[di induzione sulle grammatiche libere dal contesto]
  Dato:
  \[
    G = \{N, \Sigma, P, S\}
  \]
  \[
    \lhd \subseteq (N \cup \Sigma) \times (N \cup \Sigma)^+
  \]
  \[
    \forall A = x_1,...,x_n \in \prop, x_1 \lhd x_2 \lhd ... \lhd x_n
  \]
\end{teo}
  Dall'ultima possiamo osservare che la funghezza della stringa piu piccola e mionore della lunghezza dell'intera stringa, quindi non posso creare catente infinitamente discendenti, per questo la relazione sulle stringhe di un linguaggio si puo considerare ben fondata.
\begin{es}
  \[
    B = 0|1|B0|B1
  \]
  \[
    0 \lhd 1 \lhd B0 \lhd B1
  \]
  Osserviamo che abbiamo espresso un ordine per gli elementi 0, 1, qust'ordine per semplicita lo possiamo prenderla dalla semantica induttiva sui numeri naturali.
\end{es}
\begin{teo}[Principio d'induzione sulle grammatiche libere dal contesto]
  \[
    \forall \omega \in L(G).\prop(\omega) \Leftrightarrow
  \]
  \[
    \forall A = x_1,...,x_n \in P \;
      ([\bigwedge_{i=1}^{n} \forall \omega \in L(x_i).\prop(\omega)])
      \Rightarrow \forall \omega \in L(x_1,...,x_n).\prop(\omega))
  \]
  Per ogni produzione, le stringhe generate da ogni simbolo del lato sinistro soddisfano $\prop$ implica che le stringhe generate dall'intera produzione soddisfano $\prop$, questo è equivalente a dire che $\prop$ vale per ogni stringa appartenente al linguaggio.
\end{teo}
\begin{es}
  Cerco di dimostrare una proprieta $\prop$ su $B=0|1|B0|B1$ \\
  \[
    \forall \omega \in L(B);\omega=\omega0
  \]
  \[
    \omega' \in L(B) \Rightarrow \text{$\omega$ è un numero pari}
  \]
  \textbf{Soluzione:} \\
  Riscrittura principio d'induzione prima in maniera compatta poi in maniera esplicita
   \[
    ([\bigwedge_{i=1}^{n} \forall \omega \in L(x_i).\prop(\omega)])
    \Rightarrow \forall \omega \in L(B0).\prop(\omega))
  \]
  \[
    (\forall \omega \in L(B).\prop(\omega)])
    \Rightarrow \forall \omega \in L(B0).\prop(\omega))
  \]
  Se riesco a dimostrare che la proprietà è vera per tutte le stringhe formate da $B$, allora posso assumere che per tutte le stringhe che conterranno $B0$ la proprietà sarà vera. ($0$ è un caso base e lo diamo per vero)

\end{es}

\chapter{Definizione di Linguaggio}
TODO
\paragraph{Composizionalità:}
la proprietà per cui ciascuna stringa deve essere funzione solo dei componenti della stringa stessa.
\paragraph{Modularità:}
la proprietà per cui, aggiungendo dei costrutti ad un linguaggio $L$, non devo ridefinire la semantica dello stesso.
\begin{itemize}
	\item posso quindi definire un linguaggio in maniera incrementale, ovvero, partendo da un nucleo centrale, posso aggiungere dello "zucchero sintattico" senza modificare il nucleo di partenza.
\end{itemize}

\chapter{Semantica}
\begin{itemize}
	\item {\bf Operazionale:} COME eseguire un programma $\prop$.
	\item {\bf Denotazionale:} CHE COSA si ottiene dall'esecuzione di $\prop$.
	\item {\bf Assiomatica:} verifica SE il programma $\prop$ è corretto rispetto ad una data proprietà.
\end{itemize}
\begin{defi} dato un programma $\prop$:
	\begin{equation}
		{\ldots}z:=2; y:=z; y:=y+1; z:=y{\ldots}
	\end{equation}
	il {\bf supporto a tempo di esecuzione} è definito come:
	\begin{equation}
		[z=0, y=0] \equiv \rho(z) = 0, \rho(y) = 0
	\end{equation}
	TODO
\end{defi}

\section{Definizione formale}
\begin{defi}
	Una semantica è una quadrupla:
	\begin{equation}
	\left(L,M, \varepsilon_{i\in I}, \equiv_m\right)
	\end{equation}
	dove:
	\begin{itemize}
		\item $L$ è il linguaggio oggetto. $l$ $\in$ $L$ sono le stringhe del linguaggio a cui bisogna assegnare un significato.
		\item $M$ è il meta linguaggio, che definisce gli oggetti che si usano per assegnare la semantica (o il significato) ad $L$ $\Rightarrow \forall l \in L \exists m \in M | m(l) = \mathrm{significato}$.
		\item $\varepsilon_{i\in I}$ sono funzioni di interpretazione semantica $\Rightarrow \varepsilon_{i\in I} : L \rightarrow M $.
		\item $\equiv_m$ è l'equivalenza semantica $\subseteq M \times M$.
	\end{itemize}
\end{defi}
\begin{lemma}
	\begin{equation}
		l,l' \in L, l \equiv_l l' \Longleftrightarrow \varepsilon[l] \equiv_m \varepsilon[l']
	\end{equation}
	\begin{proof}
		L'equivalenza sul linguaggio esiste solo e soltanto se gli elementi del meta linguaggio ( $\varepsilon[l] \in M$ ) sono equivalenti.
	\end{proof}
\end{lemma}
\begin{lemma}
	\begin{equation}
		\equiv_M = \mathrm{id} \nRightarrow \equiv_L = \mathrm{id}
	\end{equation}
	\begin{proof}
		Se l'equivalenza su $M$ è l'identità ci\'o non implica la stessa proprietà per l'equivalenza sul linguaggio $L$ (?)
	\end{proof}
\end{lemma}


\section{Semantica operazionale}
\begin{defi}
	Un {\bf sistema di transizione} $T_G$ è un grafo t.c.
	$$
	T_G = ( \prop \times \rho , \longrightarrow , I , F  )
	$$
	dove
	\begin{itemize}
		\item $\prop$ sono i programmi (ad es. assegnazioni di (3.1))
		\item $\rho$ sono gli ambienti del supporto a tempo di esecuzione (ad es. (3.2))
		\item $ \longrightarrow \, \, \subseteq ( \prop , \rho ) \times ( \prop , \rho ) $
		\item $ I \subseteq ( \prop,\rho)$ iniziali
		\item $ F \subseteq (\prop, \rho)$ finali
	\end{itemize}
\end{defi}
\begin{defi}
	La {\bf semantica operazionale} è una funzione
	$$
	\mathlarger{\mathlarger{\mathlarger{\varepsilon_{op}}}} : L(G) \times \mathlarger{\mathlarger{\rho}}_{init} \longrightarrow T_G
	$$
	utilizzando la definizione formale (3.3) definiamo:
	\begin{itemize}
		\item $L$ : linguaggio oggetto (che definiamo)
		\item $M$: $<S, \longrightarrow>$ dove $S = \{ <l, \rho : \mathrm{Var}(l)\longrightarrow\mathrm{Val} >| l\in L\}$ ovvero tutti i possibili stati di tutti i possibili programmi generati da $L$, e $\longrightarrow \subseteq S \times S$
		\item $\varepsilon_{i \in I} \equiv $ definizione operativa data a inizio paragrafo.
		\item $\equiv_M$ è l'identità sui grafi (isomorfismo)
	\end{itemize}
\end{defi}

\section{Semantica denotazionale}
\begin{defi}
	La {\bf semantica denotazionale} è definita come
	$$
	\mathlarger{\mathlarger{\mathlarger{\varepsilon_{den}}}} : L(G) \times \mathlarger{\mathlarger{\rho}}_{init} \longrightarrow \mathlarger{\mathlarger{\rho}}_{fin}
	$$
	utilizzando la definizione formale (3.3):
	\begin{itemize}
		\item $L$ è il linguaggio che definiamo
		\item $M=\{ \rho : \mathrm{Var}(l) \longrightarrow \mathrm{Val} | l \in L \}$ ovvero l'insieme delle funzioni che danno valore alle variabili di $L$.
		\item $\varepsilon_{i\in I} : L\longrightarrow M , l \longrightarrow \mathrm{val}$
		\item $\equiv_m$ è l'identità
	\end{itemize}
\end{defi}
Rappresenta solo il risultato dell'esecuzione del programma, ovvero l'insieme degli ambienti finali.

\section{Semantica assiomatica}
\begin{defi}
	Una {\bf regola di inferenza} stabilisce che:
	$$
	\frac{p_1 \wedge \ldots \wedge p_n}{C}
	$$
	se $ p_1 \wedge \ldots \wedge p_n $ sono vere allora $C$ è vera.
\end{defi}
\begin{defi}
	Ho un {\bf assioma} se:
	$$
	\frac[20pt]{}{C}
	$$
	ovvero se non ho proprietà su cui basare la conclusione.
\end{defi}
\begin{defi}
	La {\bf semantica assiomatica} è definita sulla base di regole di inferenza ed assiomi, e dimostra la verdicità di un programma costruendo un albero le cui foglie sono tutte {\bf assiomi}.
	\\
	Utilizzando la definizione formale di semantica a (3.3):
	\begin{itemize}
		\item $L$ è il linguaggio che definiamo
		\item $M$ è l'insieme di alberi di derivazione per $l\in L$
		\item $\varepsilon_{i\in I}$ sono le interpretazioni degli assiomi e delle regole di inferenza
		\item $\equiv_M$ è l'identità sugli alberi di derivazione
	\end{itemize}
\end{defi}

\section{Contesto}
\begin{defi}
	Un {\bf contesto} $C[\cdot]$ per un programma è un programma con un "buco" (una parte non specificata).
	\end{defi}
	\begin{enumerate}
	\item {\bf Principio di Subsitutivity:}
	\\
	$l, l' \in L$ che possono comparire nello stesso contesto $C[\cdot]$:
	$$
		l \equiv_L l' \Rightarrow \forall C[\cdot] , C[l] \equiv_L C[l']
	$$
	\item {\bf Principio di Full abstraction:}
	$$
		l,l' \in L, \forall C[\cdot], C[l] \equiv_L C[l'] \Longrightarrow l \equiv_L l'
	$$
\end{enumerate}

\chapter{Linguaggio imperativo: IMP}
\begin{defi}
  La memoria è rappresentata da una funzione:
  \begin{equation}
    \sigma:\bigcup_{\tau \in \mathrm{STyp}} \mathrm{Loc}_\tau \rightarrow \mathrm{SVal} \cup \{?,\bot\}
  \end{equation}
  Dove $?$ rappresenta una locazione di memoria allocata e non inizializzata,
  mentre $\bot$ rappresenta una locazione non allocata.\\
  Si suppone per comodità che $\forall \tau.\lvert\mathrm{Loc}_\tau\rvert = \infty$,
  ovvero che la memoria sia infinita.
\end{defi}
\begin{defi}
  Si definisce l'estensione di $\sigma$ (definita su $L$) a $\sigma_0$ (definita su $L_0$)
  \begin{equation}
    \sigma[\sigma_0](l)=
    \begin{cases}
      \sigma_0(l) & \text{se } l \in L_0 \\
      \sigma(l) & \text{se } l \in (L \setminus L_0)
    \end{cases}
  \end{equation}
\end{defi}
TODO: ambienti statici e dinamici e loro estensioni, compatibilità di amb.
\section{Espressioni}
Il linguaggio oggetto $l$ delle espressioni è generato dalla grammatica
\[E \mathrel{::=} k\ | \mathrm{\ id\ } |\ E_0 \bop E_1\ |\ \uop E \]
\subsection{Semantica statica}
\begin{equation}
  \emptyset\vdash_I n:\text{int}
  \label{expr_stat_const_int}
\end{equation}
\begin{equation}
  \emptyset \vdash (tt|ff):\text{bool}
  \label{expr_stat_const_bool}
\end{equation}
\begin{equation}
  \Delta \vdash \mathrm{id}:\Delta(\mathrm{id})
  \label{expr_stat_id}
\end{equation}
\begin{equation}
  \frac{
    \Delta \vdash E_0:\tau_0, \Delta \vdash E_1:\tau_1
  }{
    \Delta \vdash E_0 \bop E_1:\tau_{\bop}(\tau_0,\tau_1)
  }
  \label{expr_stat_bop}
\end{equation}
\begin{equation}
  \frac{
    \Delta \vdash E : \tau
  }{
    \Delta \vdash \uop E : \tau_{\uop}(\tau)
  }
  \label{expr_stat_uop}
\end{equation}
\subsection{Semantica dinamica}
Il metalinguaggio $m$ è un grafo (sistema di transizione)
\[(<\prop, \sigma>,\quad \mathbin{\rightarrow}_e \subseteq <\prop, \sigma> \times <\prop, \sigma>)\]
Si procede per induzione sulla struttura della sintassi.\\
I {\bf casi base} sono $k$ e id. $k$ è di per s\'e una configurazione finale.
Per id si ha l'assioma:
\begin{equation}
  \frac{}{\rho \vdash <\mathrm{id}, \sigma> \mathbin{\rightarrow} <k,\sigma>, k=\sigma(\rho(\mathrm{id}))}
\end{equation}
Per $E_0 \bop E_1$ le ipotesi induttive sono $E_0, E_1$ e si hanno le regole:
\begin{equation}
  \frac{
    \rho \vdash <E_0,\sigma> \mathbin{\rightarrow} <E_0',\sigma>
  }{
    \rho \vdash <E_0 \bop E_1,\sigma> \mathbin{\rightarrow} <E_0' \bop E_1,\sigma>
  }
\end{equation}
\begin{equation}
  \frac{
    \rho \vdash <E_1,\sigma> \mathbin{\rightarrow} <E_1',\sigma>
  }{
    \rho \vdash <k_0 \bop E_1,\sigma> \mathbin{\rightarrow} <k_0 \bop E_1',\sigma>
  }
\end{equation}
\begin{equation}
  \frac{}{
  \rho \vdash <k_0 \bop k_1,\sigma> \mathbin{\rightarrow} <k',\sigma>
  }
\end{equation}
dove $k'$ è il risultato di $k_0 \bop k_1$.\\
Per $\uop E$, l'ipotesi induttiva è $E$ e si hanno le regole:
\begin{equation}
  \frac{
    \rho \vdash <E_0,\sigma> \mathbin{\rightarrow} <E_0',\sigma>
  }{
    \rho \vdash <\uop E_0,\sigma> \mathbin{\rightarrow} <\uop E_0',\sigma>
  }
\end{equation}
\begin{equation}
  \frac{}{
  \rho \vdash <\uop k,\sigma> \mathbin{\rightarrow} <k',\sigma>
  }
\end{equation}
dove $k'$ è il risultato di $\uop k_0$.

TODO: esempio, nota sull'ordine di interpretazione
\section{Bound Identifiers, Free Identifiers}
Un nome, all'interno di un programma, può apparire in posizione:
\begin{description}
  \item[di definizione] quando è definito
  \item[legata] quando appare dopo essere stato definito, cioè si può
    determinare la definizione corrispondente
  \item[libera] quando appare senza essere stato definito
\end{description}
I nomi in posizione di definizione e legata sono detti {\em bound identifiers}.
\\ I nomi in posizione libera sono detti {\em free identifiers}.
\begin{defi}
  Si definisce la funzione
  \[\FI: \IDE \rightarrow \IDE\]
  che restituisce l'insieme dei free isentifiers, e la funzione
  \[\BI: \IDE \rightarrow \IDE\]
  che restituisce l'insieme dei bound identifiers. Vale:
  \[\forall \mathrm{prg}.\FI(\mathrm{prg})\cap \BI(\mathrm{prg})=\emptyset\]
\end{defi}
\begin{lemma}
  Se in un'espressione sono presenti identificatori liberi non già definiti
  altrove, l'espressione non può essere valutata.
  \[\FI(E)\not\subseteq I \Rightarrow \Delta\not\vdash_I E:\tau\]
  con $I=\{\mathrm{id}|\exists(\mathrm{id},\tau)\in\Delta\}$.
  \begin{proof}
    La premessa è equivalente ad affermare che
    \[\exists x . x \in \FI(E) \wedge x \not\in I\]
    Si dimostra per induzione sulla struttura della sintassi.

    Per il caso base $k$, si ha $\FI(k)=\emptyset \subseteq I \forall I$.

    Per il caso base id, si ha $\FI(\mathrm{id}) = \{\mathrm{id}\},
      \mathrm{id} \not\in I$.
    Da ciò $\Delta(\mathrm{id}) = \bot \Rightarrow \Delta \not\vdash_I \mathrm{id}:\tau$.

    Per il passo induttivo $E_0 \bop E_1$, si ha che 
    \[\exists x \in \FI(E_0 \bop E_1) \wedge x \not\in I\]
    o anche, per la definizione di $\FI$
    \[(\exists x \in \FI(E_0) \wedge x \not\in I) \vee (\exists x \in \FI(E_1) \wedge x \not\in I)\]
    Nel primo caso, ciò equivale a dire che $\FI(E_0) \not\subseteq I
    \perhpind
    \Delta \not\vdash_I E_0 : \tau_0 \Rightarrow $ non vale la \ref{expr_stat_bop}.
    Il secondo caso è simmetrico.

    Per il passo induttivo $\uop E$, si ha che
    \[\exists x \in \FI(\uop E)\]
    o anche, per la definizione di $\FI$
    \[\exists x \in \FI(E) \wedge x \not\in I
      \perhpind
    \Delta \not\vdash_I E : \tau\]
    quindi non vale la \ref{expr_stat_uop}.
  \end{proof}
\end{lemma}
\end{document}
