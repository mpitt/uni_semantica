%        File: appunti.tex
%     Created: Mon Feb 24 02:00 PM 2014 C
% Last Change: Mon Feb 24 02:00 PM 2014 C
%
\documentclass[a4paper]{report}
\usepackage[utf8]{inputenc}
\usepackage[italian]{babel}
\usepackage{amsmath}
\usepackage{amssymb}
\usepackage{amsthm}

\newcommand{\bnfn}{\ \mathrm{b.f.}\ }
\newcommand{\sse}{se e solo se}
\newcommand{\tc}{\ \mathrm{t.c.}\ }
\newcommand{\prop}{\mathcal{P}}
\newcommand{\base}[1]{\mathtt{base}_{#1}}

\theoremstyle{definition} \newtheorem*{defi}{Def}
\theoremstyle{plain} \newtheorem{lemma}{Lemma}
\theoremstyle{plain} \newtheorem{teo}{Teorema}
\theoremstyle{remark} \newtheorem*{es}{Esempio}

\begin{document}
\chapter{Induzione strutturale}
\section*{Introduzione}
Sia $A$ un insieme, sia $\lhd$ una relazione binaria definita su $A$
($\lhd : A \times A$).
\begin{defi}
  $\lhd$ è $\bnfn$ (ben fondata) se $\nexists \ldots \lhd a_i \lhd \ldots \lhd a_1 \lhd a_0$
  \\
  ovvero non esistono catene infinite discendenti.
  \\
  $(A, \lhd)$ è un insieme ben fondato se $\lhd$ è $\bnfn$
\end{defi}
Sia $\unlhd$ la chiusura riflessiva e transitiva di $\lhd \ \bnfn$ su $A$.
\begin{lemma}
  $\unlhd$ non è mai $\bnfn$
\end{lemma}
\begin{proof}
  Sia $a_i \in A$. $\exists \ldots a_i \unlhd a_i \unlhd \ldots$ \qedhere
\end{proof}
\begin{defi}
  Sia $a \in A$. $a$ è \emph{minimale} in $A$ rispetto a $\lhd$ se
  $\forall b \lhd a, b \not\in A$.
\end{defi}
\begin{lemma}
  \label{lemma:2}
  $\lhd$ è ben fondato su $A$ \sse ogni $B \subseteq A$ ha un elemento
  minimale rispetto a $\lhd$.
\end{lemma}
\begin{proof}
  \begin{itemize}
    \item{$\Rightarrow$)} Da $ B \subseteq A$ e $(A, \lhd) \bnfn$ si ha che
      non esiste $\ldots \lhd b_i \lhd \ldots \lhd b_0$, quindi 
      $\exists b_n \tc b_n \lhd \ldots \lhd b_i \lhd \ldots \lhd b_0$
      ovvero $b_n$ è minimale in $B$ rispetto a $\lhd$.
    \item{$\Leftarrow$)} Per assurdo. Supponiamo che esista
      $\ldots \lhd a_i \lhd \ldots \lhd a_0$, ovvero $\neg (\lhd \bnfn)$.
      Allora l'insieme $B=\{a_i|i\in\mathbb{N}\}$ (insieme degli elementi
      della sequenza) non ha un minimale, cosa che contraddice l'ipotesi.
      Quindi una tale sequenza non esiste, ovvero $\lhd \bnfn$
  \end{itemize}
\end{proof}

\section{Principio di induzione noetheriana}
\begin{teo}[Principio di induzione noetheriana (prima forma)]
  Sia $\prop$ una proprietà su $(A, \lhd) \bnfn$.\\
  \[
    \forall a \in A.\prop(a) \Leftrightarrow 
    \forall a \in A.([\forall b \lhd a . \prop(b)] 
    \Rightarrow \prop(a))
  \]
\end{teo}
\begin{proof}
  \begin{itemize}
    \item{$\Rightarrow$)} Ovvia.
    \item{$\Leftarrow$)} Per assurdo.\\
      Supponiamo
      \begin{equation}
        \forall a \in A.([\forall b \lhd a.\prop(b)]
        \Leftarrow \prop(a))
      \end{equation}
      e
      \begin{equation}
        \exists c \in A .\neg\prop(c)
      \end{equation}
      Sia $C=\{c \in A | \neg \prop(c)\} \subseteq A$.\\
      Per il lemma \ref{lemma:2}, $\exists \hat{c}$ minimale di $C$ rispetto a 
      $\lhd$, allora $\neg \prop(\hat{c})$\\
      Per $\hat{c}$ minimale, $\forall b \lhd \hat{c}.b \not\in C$, allora
      $\prop(b)$, ovvero per (1.1) $\prop(\hat{c})$
  \end{itemize}
\end{proof}

\begin{defi}
  $\base{A}=\{a \in A | a$ è minimale in $A$ rispetto a $\lhd \}$.\\
  Osserviamo che $\forall a \in \base{A}, \forall b \in A . b \ntriangleleft a$.
\end{defi}

\begin{teo}[Principio di induzione noetheriana - seconda forma]
  Sia $\prop$ una proprietà su $(A, \lhd) \bnfn$.\\
  \[
    \forall a \in A.\prop(a) \Leftrightarrow 
    \begin{pmatrix} 
      \forall a \in \base{A}.\prop(a)\\
      \wedge\\
      \forall a \in (A \setminus \base{A})
      .([\forall b \lhd a . \prop(b)] \Rightarrow \prop(a))
    \end{pmatrix}
  \]
\end{teo}

\begin{teo}[Induzione matematica]
  Sia $A=\mathbb{N}$.\\
  Sia $n \lhd m \Leftrightarrow m=n+1 \quad n,m \in \mathbb{N}$.\\
  Osserviamo che $\lhd$ è ben fondata: $0\lhd1\lhd2\lhd\dots$.\\
  Osserviamo $\base{\mathbb{N}}=\{0\}$.
  \[
    \forall m \in \mathbb{N}.\prop(m) \Leftrightarrow 
    \begin{pmatrix} 
      \prop(0)\\
      \wedge\\
      \forall m > 0
      .(\prop(m-1) \Rightarrow \prop(m))
    \end{pmatrix}
  \]
\end{teo}

\begin{defi}
  $a^\lhd=\{b \in A|b \lhd a\}$ con $(A,\lhd)\bnfn$.
\end{defi}
\begin{defi}
  Sia $f: A \rightarrow B$, sia $A' \subseteq A$.
  $f_{|A'}=\{(a, f(a))|a \in A'\}$.
\end{defi}

\begin{teo}[di ricorsione / delle definizione noetheriane]
  Sia $(A,\lhd)\bnfn$
  \[
    \forall a \in A, \forall h:a^\lhd \rightarrow B.
    F(a,h) \in B
  \]
  ($F$ è detta \emph{operatore di composizione}). Allora
  \[
    \exists! f: A \rightarrow B \tc
    \forall a \in A.f(a)=F(a,f_{|a^\lhd})
  \]
\end{teo}
\begin{es}
  Sia $A=\mathbb{N}$, sia $n \lhd m \Leftrightarrow m=n+1$.\\
  \[
    \mathtt{Fact}(n)=
    \begin{cases}
      1 & \text{se } n=0\\
      n \cdot \mathtt{Fact}(n-1) & \text{se } n>0
    \end{cases}
  \]
  In questo caso, $f = \mathtt{Fact}$ e $F$ è la moltiplicazione.\\
  $f(n)=F(n,f(n-1))=n \cdot f(n-1)$
\end{es}

\end{document}


