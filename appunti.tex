%        File: appunti.tex
%     Created: Mon Feb 24 02:00 PM 2014 C
% Last Change: Mon Feb 24 02:00 PM 2014 C
%
\documentclass[a4paper]{report}
\usepackage[utf8]{inputenc}
\usepackage[italian]{babel}
\usepackage{amsmath}
\usepackage{amssymb}
\usepackage{amsthm}
\usepackage{relsize}

\newcommand{\bnfn}{\ \mathrm{b.f.}\ }
\newcommand{\sse}{se e solo se}
\newcommand{\tc}{\ \mathrm{t.c.}\ }
\newcommand{\prop}{\mathcal{P}}
\newcommand{\base}[1]{\mathtt{base}_{#1}}

\theoremstyle{definition} \newtheorem*{defi}{Def}
\theoremstyle{plain} \newtheorem{lemma}{Lemma}
\theoremstyle{plain} \newtheorem{teo}{Teorema}
\theoremstyle{remark} \newtheorem*{es}{Esempio}

\DeclareRobustCommand{\frac}[3][0pt]{%
  {\begingroup\hspace{#1}#2\hspace{#1}\endgroup\over\hspace{#1}#3\hspace{#1}}}

\begin{document}
\chapter{Induzione strutturale}
\section*{Introduzione}
Sia $A$ un insieme, sia $\lhd$ una relazione binaria definita su $A$
($\lhd : A \times A$).
\begin{defi}
  $\lhd$ è $\bnfn$ (ben fondata) se $\nexists \ldots \lhd a_i \lhd \ldots \lhd a_1 \lhd a_0$
  \\
  ovvero non esistono catene infinite discendenti.
  \\
  $(A, \lhd)$ è un insieme ben fondato se $\lhd$ è $\bnfn$
\end{defi}
Sia $\unlhd$ la chiusura riflessiva e transitiva di $\lhd \ \bnfn$ su $A$.
\begin{lemma}
  $\unlhd$ non è mai $\bnfn$
\end{lemma}
\begin{proof}
  Sia $a_i \in A$. $\exists \ldots a_i \unlhd a_i \unlhd \ldots$ \qedhere
\end{proof}
\begin{defi}
  Sia $a \in A$. $a$ è \emph{minimale} in $A$ rispetto a $\lhd$ se
  $\forall b \lhd a, b \not\in A$.
\end{defi}
\begin{lemma}
  \label{lemma:2}
  $\lhd$ è ben fondato su $A$ \sse{} ogni $B \subseteq A$ ha un elemento
  minimale rispetto a $\lhd$.
\end{lemma}
\begin{proof}
  \begin{itemize}
    \item{$\Rightarrow$)} Da $ B \subseteq A$ e $(A, \lhd) \bnfn$ si ha che
      non esiste $\ldots \lhd b_i \lhd \ldots \lhd b_0$, quindi
      $\exists b_n \tc b_n \lhd \ldots \lhd b_i \lhd \ldots \lhd b_0$
      ovvero $b_n$ è minimale in $B$ rispetto a $\lhd$.
    \item{$\Leftarrow$)} Per assurdo. Supponiamo che esista
      $\ldots \lhd a_i \lhd \ldots \lhd a_0$, ovvero $\neg (\lhd \bnfn)$.
      Allora l'insieme $B=\{a_i|i\in\mathbb{N}\}$ (insieme degli elementi
      della sequenza) non ha un minimale, cosa che contraddice l'ipotesi.
      Quindi una tale sequenza non esiste, ovvero $\lhd \bnfn$
  \end{itemize}
\end{proof}
\newpage

\section{Principio di induzione noetheriana}
\begin{teo}[Principio di induzione noetheriana (prima forma)]
  Sia $\prop$ una proprietà su $(A, \lhd) \bnfn$.\\
  \[
    \forall a \in A.\prop(a) \Leftrightarrow
    \forall a \in A.([\forall b \lhd a . \prop(b)]
    \Rightarrow \prop(a))
  \]
\end{teo}
\begin{proof}
  \begin{itemize}
    \item{$\Rightarrow$)} Ovvia.
    \item{$\Leftarrow$)} Per assurdo.\\
      Supponiamo
      \begin{equation}
        \forall a \in A.([\forall b \lhd a.\prop(b)]
        \Leftarrow \prop(a))
      \end{equation}
      e
      \begin{equation}
        \exists c \in A .\neg\prop(c)
      \end{equation}
      Sia $C=\{c \in A | \neg \prop(c)\} \subseteq A$.\\
      Per il lemma \ref{lemma:2}, $\exists \hat{c}$ minimale di $C$ rispetto a
      $\lhd$, allora $\neg \prop(\hat{c})$\\
      Per $\hat{c}$ minimale, $\forall b \lhd \hat{c}.b \not\in C$, allora
      $\prop(b)$, ovvero per (1.1) $\prop(\hat{c})$
  \end{itemize}
\end{proof}

\begin{defi}
  $\base{A}=\{a \in A | a$ è minimale in $A$ rispetto a $\lhd \}$.\\
  Osserviamo che $\forall a \in \base{A}, \forall b \in A . b \ntriangleleft a$.
\end{defi}

\begin{teo}[Principio di induzione noetheriana - seconda forma]
  Sia $\prop$ una proprietà su $(A, \lhd) \bnfn$.\\
  \[
    \forall a \in A.\prop(a) \Leftrightarrow
    \begin{pmatrix}
      \forall a \in \base{A}.\prop(a)\\
      \wedge\\
      \forall a \in (A \setminus \base{A})
      .([\forall b \lhd a . \prop(b)] \Rightarrow \prop(a))
    \end{pmatrix}
  \]
\end{teo}

\begin{teo}[Induzione matematica]
  Sia $A=\mathbb{N}$.\\
  Sia $n \lhd m \Leftrightarrow m=n+1 \quad n,m \in \mathbb{N}$.\\
  Osserviamo che $\lhd$ è ben fondata: $0\lhd1\lhd2\lhd\dots$.\\
  Osserviamo $\base{\mathbb{N}}=\{0\}$.
  \[
    \forall m \in \mathbb{N}.\prop(m) \Leftrightarrow
    \begin{pmatrix}
      \prop(0)\\
      \wedge\\
      \forall m > 0
      .(\prop(m-1) \Rightarrow \prop(m))
    \end{pmatrix}
  \]
\end{teo}

\begin{defi}
  $a^\lhd=\{b \in A|b \lhd a\}$ con $(A,\lhd)\bnfn$.
\end{defi}
\begin{defi}
  Sia $f: A \rightarrow B$, sia $A' \subseteq A$.
  $f_{|A'}=\{(a, f(a))|a \in A'\}$.
\end{defi}

\begin{teo}[di ricorsione / delle definizione noetheriane]
  Sia $(A,\lhd)\bnfn$
  \[
    \forall a \in A, \forall h:a^\lhd \rightarrow B.
    F(a,h) \in B
  \]
  ($F$ è detta \emph{operatore di composizione}). Allora
  \[
    \exists! f: A \rightarrow B \tc
    \forall a \in A.f(a)=F(a,f_{|a^\lhd})
  \]
\end{teo}
\begin{es}
  Sia $A=\mathbb{N}$, sia $n \lhd m \Leftrightarrow m=n+1$.\\
  \[
    \mathtt{Fact}(n)=
    \begin{cases}
      1 & \text{se } n=0\\
      n \cdot \mathtt{Fact}(n-1) & \text{se } n>0
    \end{cases}
  \]
  In questo caso, $f = \mathtt{Fact}$ e $F$ è la moltiplicazione.\\
  $f(n)=F(n,f(n-1))=n \cdot f(n-1)$
\end{es}
\newpage{}
\begin{teo}[di induzione sulle grammatiche libere dal contesto]
  Dato:
  \[
    G = \{N, \Sigma, P, S\}
  \]
  \[
    \lhd \subseteq (N \cup \Sigma) \times (N \cup \Sigma)^+
  \]
  \[
    \forall A = x_1,...,x_n \in \prop, x_1 \lhd x_2 \lhd ... \lhd x_n
  \]
\end{teo}
  Dall'ultima possiamo osservare che la funghezza della stringa piu piccola e mionore della lunghezza dell'intera stringa, quindi non posso creare catente infinitamente discendenti, per questo la relazione sulle stringhe di un linguaggio si puo considerare ben fondata.
\begin{es}
  \[
    B = 0|1|B0|B1
  \]
  \[
    0 \lhd 1 \lhd B0 \lhd B1
  \]
  Osserviamo che abbiamo espresso un ordine per gli elementi 0, 1, qust'ordine per semplicita lo possiamo prenderla dalla semantica induttiva sui numeri naturali.
\end{es}
\begin{teo}[Principio d'induzione sulle grammatiche libere dal contesto]
  \[
    \forall \omega \in L(G).\prop(\omega) \Leftrightarrow
  \]
  \[
    \forall A = x_1,...,x_n \in P \;
      ([\bigwedge_{i=1}^{n} \forall \omega \in L(x_i).\prop(\omega)])
      \Rightarrow \forall \omega \in L(x_1,...,x_n).\prop(\omega))
  \]
  Per ogni produzione, le stringhe generate da ogni simbolo del lato sinistro soddisfano $\prop$ implica che le stringhe generate dall'intera produzione soddisfano $\prop$, questo è equivalente a dire che $\prop$ vale per ogni stringa appartenente al linguaggio.
\end{teo}
\begin{es}
  Cerco di dimostrare una proprieta $\prop$ su $B=0|1|B0|B1$ \\
  \[
    \forall \omega \in L(B);\omega=\omega0
  \]
  \[
    \omega' \in L(B) \Rightarrow \text{$\omega$ è un numero pari}
  \]
  \textbf{Soluzione:} \\
  Riscrittura principio d'induzione prima in maniera compatta poi in maniera esplicita
   \[
    ([\bigwedge_{i=1}^{n} \forall \omega \in L(x_i).\prop(\omega)])
    \Rightarrow \forall \omega \in L(B0).\prop(\omega))
  \]
  \[
    (\forall \omega \in L(B).\prop(\omega)])
    \Rightarrow \forall \omega \in L(B0).\prop(\omega))
  \]
  Se riesco a dimostrare che la proprietà è vera per tutte le stringhe formate da $B$, allora posso assumere che per tutte le stringhe che conterranno $B0$ la proprietà sarà vera. ($0$ è un caso base e lo diamo per vero)

\end{es}

\chapter{Definizione di Linguaggio}
TODO
\paragraph{Composizionalit\'a:}
la propriet\'a per cui ciascuna stringa deve essere funzione solo dei componenti della stringa stessa.
\paragraph{Modularit\'a:}
la propriet\'a per cui, aggiungendo dei costrutti ad un linguaggio $L$, non devo ridefinire la semantica dello stesso.
\begin{itemize}
\item posso quindi definire un linguaggio in maniera incrementale, ovvero, partendo da un nucleo centrale, posso aggiungere dello "zucchero sintattico" senza modificare il nucleo di partenza.
\end{itemize}

\chapter{Semantica}
\begin{itemize}
\item {\bf Operazionale:} COME eseguire un programma $P$.
\item {\bf Denotazionale:} CHE COSA si ottiene dall'esecuzione di $P$.
\item {\bf Assiomatica:} verifica SE il programma \'e corretto rispetto ad una data propriet\'a.
\end{itemize}

\begin{defi} dato un programma $P$:
\begin{equation}
{\ldots}z:=2; y:=z; y:=y+1; z:=y{\ldots}
\end{equation}
il {\bf supporto a tempo di esecuzione} \'e definito come:
\begin{equation}
[z=0, y=0] \equiv \rho(z) = 0, \rho(y) = 0
\end{equation}
TODO
\end{defi}
\newpage
\section{Semantica operazionale}
\begin{defi}
Un {\bf sistema di transizione} $T_G$ \'e un grafo t.c.
$$
T_G = ( P \times \rho , \longrightarrow , I , F  )
$$
dove
\begin{itemize}
\item $P$ sono i programmi (ad es. assegnazioni di (3.1))
\item $\rho$ sono gli ambienti del supporto a tempo di esecuzione (ad es. (3.2))
\item $ \longrightarrow \, \, \subseteq ( P , \rho ) \times ( P , \rho ) $
\item $ I \subseteq ( P,\rho)$ iniziali
\item $ F \subseteq (P, \rho)$ finali
\end{itemize}
\end{defi}

\begin{defi}
La {\bf semantica operazionale} \'e una funzione
$$
\mathlarger{\mathlarger{\mathlarger{\varepsilon_{op}}}} : L(G) \times \mathlarger{\mathlarger{\rho}}_{init} \longrightarrow T_G
$$
\end{defi}

\section{Semantica denotazionale}
\begin{defi}
La {\bf semantica denotazionale} \'e definita come
$$
\mathlarger{\mathlarger{\mathlarger{\varepsilon_{den}}}} : L(G) \times \mathlarger{\mathlarger{\rho}}_{init} \longrightarrow \mathlarger{\mathlarger{\rho}}_{fin}
$$
\end{defi}
Rappresenta solo il risultato dell'esecuzione del programma, ovvero l'insieme degli ambienti finali.

\section{Semantica assiomatica}
\begin{defi}
Una {\bf regola di inferenza} stabilisce che:
$$
\frac{p_1 \wedge \ldots \wedge p_n}{C}
$$
se $ p_1 \wedge \ldots \wedge p_n $ sono vere allora $C$ \'e vera.
\end{defi}
\begin{defi}
Ho un {\bf assioma} se:
$$
\frac[20pt]{}{C}
$$
ovvero se non ho propriet\'a su cui basare la conclusione.
\end{defi}
\begin{defi}
La {\bf semantica assiomatica} \'e definita sulla base di regole di inferenza ed assiomi, e dimostra la verdicit\'a di un programma costruendo un albero le cui foglie sono tutte {\bf assiomi}.
\end{defi}
\end{document}
